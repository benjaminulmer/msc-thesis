\makethesistitle
\pagenumbering{roman}     % resets page counter to one
\setcounter{page}{2}
%\chapter*{UNIVERSITY OF CALGARY \\ FACULTY OF GRADUATE STUDIES}
%\thispagestyle{empty}
%The undersigned certify that they have read, and recommend
%to the Faculty of Graduate Studies for acceptance, a \Thesis\ entitled
%``\thesistitle'' submitted by \Author\
%in partial fulfillment of the requirements for the degree of
%\Degree.\\

%
%    Substitute  List of Examiners
%
%\begin{signing}{Department of Academic Computing}
%\signline
%Chairman, Dr.~John D.~Doe \\
%Department of Academic Computing \\
%Services  \\
%\signline
%Chairman, Dr.~John D.~Doe \\
%Department of Academic Computing \\
%Services  \\
%\signline
%Chairman, Dr.~John D.~Doe \\
%Department of Academic Computing \\
%Services  \\
%\signline
%Chairman, Dr.~John D.~Doe \\
%Department of Academic Computing \\
%Services  \\
%\newsigncolumn         use this command to start a new column if necessary
%\newsigncolumn
%\signline
%Chairman, Dr.~John D.~Doe \\
%Department of Academic Computing \\
%Services  \\
%\signline
%Dr.~Jane Smith \\
%Department of Academic Computing  \\

%\signline
%Dr.~A.~B.~Brown \\
%Department of Academic Computing  \\
%\end{signing}
%
\newpage
\phantomsection
\altchapter{\bf{Abstract}}
Recent technological advancements have led to unprecedented amounts of 3D data becoming available about the Earth.
With this, technologies that facilitate the efficient integration and management of geospatial data are becoming increasingly important.
Hierarchical partitionings of the surface of the Earth, known as Discrete Global Grid Systems (DGGS), have proven to be useful tools for integrating data on the Earth's surface; however, they have no native support for 3D data.
Instead, a 3D version of this data structure is needed.
An Earth-centric 3D DGGS that respects the spherical nature of the planet is desirable, but this approach introduces the problem of reduced cell size and compactness near the centre of the grid.
In this thesis, we explore a particular class of refinement methods, which we term semiregular degenerate, as a potential solution to these issues.
We propose both a modification of an existing 3D DGGS to improve its volume preservation properties and a general framework for extending any existing DGGS to the third dimension.
We also derive a set of mapping functions that facilitate efficient encoding and decoding algorithms for both these methods.
Grid properties and algorithm runtimes are evaluated quantitatively, and a series of use cases are used to evaluate the grid extension methodology by creating a 3D DGGS explicitly tailored for each example application.

\newpage
\phantomsection
\altchapter{\bf{Acknowledgements}}
Placeholder for acknowledgements.

\begin{singlespace}
\newpage
\phantomsection
\tableofcontents
\pagestyle{plain}
\newpage
\phantomsection
\listoftables
\pagestyle{plain}
\newpage
\phantomsection
\listoffigures
\pagestyle{plain}
\clearpage
\clearpage          % otherwise tables will be numbered wrong
\end{singlespace}
