\makethesistitle
\pagenumbering{roman}     % resets page counter to one
\setcounter{page}{2}
%\chapter*{UNIVERSITY OF CALGARY \\ FACULTY OF GRADUATE STUDIES}
%\thispagestyle{empty}
%The undersigned certify that they have read, and recommend
%to the Faculty of Graduate Studies for acceptance, a \Thesis\ entitled
%``\thesistitle'' submitted by \Author\
%in partial fulfillment of the requirements for the degree of
%\Degree.\\

%
%    Substitute  List of Examiners
%
%\begin{signing}{Department of Academic Computing}
%\signline
%Chairman, Dr.~John D.~Doe \\
%Department of Academic Computing \\
%Services  \\
%\signline
%Chairman, Dr.~John D.~Doe \\
%Department of Academic Computing \\
%Services  \\
%\signline
%Chairman, Dr.~John D.~Doe \\
%Department of Academic Computing \\
%Services  \\
%\signline
%Chairman, Dr.~John D.~Doe \\
%Department of Academic Computing \\
%Services  \\
%\newsigncolumn         use this command to start a new column if necessary
%\newsigncolumn
%\signline
%Chairman, Dr.~John D.~Doe \\
%Department of Academic Computing \\
%Services  \\
%\signline
%Dr.~Jane Smith \\
%Department of Academic Computing  \\

%\signline
%Dr.~A.~B.~Brown \\
%Department of Academic Computing  \\
%\end{signing}
%
\newpage
\phantomsection
\altchapter{\bf{Abstract}}
\pagestyle{plain}
Recent technological advancements have led to unprecedented amounts of 3D data becoming available about the Earth.
With this, technologies that facilitate the efficient integration and management of geospatial data are becoming increasingly important.
Hierarchical partitionings of the surface of the Earth, known as Discrete Global Grid Systems (DGGS), have proven to be useful tools for integrating data on the Earth's surface; however, they have no native support for 3D data.
Instead, a 3D version of this data structure is needed.
An Earth-centric 3D DGGS that respects the spherical nature of the planet is desirable, but this approach introduces the problem of reduced cell size and compactness near the centre of the grid.
In this thesis, we explore a particular class of refinement methods, which we term semiregular degenerate, as a potential solution to these issues.
We propose both a modification of an existing 3D DGGS to improve its volume preservation properties and a general framework for extending any existing DGGS to the third dimension.
We also derive a set of mapping functions that facilitate efficient encoding and decoding algorithms for both these methods.
Grid properties and algorithm runtimes are evaluated quantitatively, and a series of use cases are used to evaluate the grid extension methodology by creating a 3D DGGS explicitly tailored for each example application.


\newpage
\phantomsection
\altchapter{\bf{Acknowledgements}}
\pagestyle{plain}
This document owes its existence to countless individuals.


First and foremost, none of this would have been possible without the guidance of my incredible supervisor, Dr. Faramarz Samavati.
I would have never even \textit{started} the journey of grad school if it hadn't been for the 1-minute pitch you gave me when I was a student in your modelling class.
Since then, you have helped me navigate many unfamiliar waters, and I surely would not have made it to the end without your support and supervision.
You taught me many valuable lessons over the years, including what makes a great presentation, how to think like a scientist, and the importance of understanding the geometric intuition of things.


I would also like to thank the organizations that put a roof over my head and food on my plate throughout this degree.
Thank you to NSERC, the Alberta Government, the University of Calgary, and Global Grid Systems for the funding they provided.
Additional thanks to Idan Shatz and Dallas Rathbone for the countless discussions on DGGS and related topics.


If you had told me three years ago I would have to defend my thesis over Zoom, I would have probably said, ``...what's Zoom?''
A special thanks to my examiners, Dr. P. and Dr. Emmanuel Stefanakis, for reading my thesis and providing feedback during these unusual circumstances.
I'm thankful to live in a time where technology allowed my defence to continue while everyone remained safe and isolated.


I will forever be grateful for the support and friendship of everyone on the 6th floor of Math Sciences.
To the labmates who have since moved on---Troy Alderson, Kamyar Allahverdi, Kathleen Ang, John Brosz, Shima Dadkhahfard, Tim Davison, Mohammed Elbaz, Erika Harrison, Mahmudul Hasan, Roya Olyazadeh, Samin Sabokrohiyeh, Mark Sherlock, Ben Stewart, and Gaurav Tripathi---thank you for the advice and feedback you gave while our journeys overlapped.
To those that remain---Majid Amirfakhrian, Hessam Djavaherpour, Katy Etemad, John Hall, Mohammad Hameed, Meysam Kazemi, Hooman Khosravi, Mia MacTavish, Amir Mirtabatabaeipour, Christopher Mossman, Mahsa Qadimzadeh, Xi Wang, Lakin Wecker, and Fatemeh Yazdanbakhsh---thank you for all the good times and everything you've done to support me.
Similar thanks go to the extended graphics family---Cory Bloor, Sonny Chan, Mik Cieslak, Pascal Ferraro, Alex Garcia, Philmo Gu, Jeremy Hart, Haysn Hornbeck, Desmond Larsen-Rosner, Andrew Owens, Dr. P., Lee Ringham, Allan Rocha, Surbhi Sachan, and Erik Spooner.
Thank you to the past and current members of the iLab DnD group---Neil Chulpong-\ satorn, Kat Currier, Kurtis Danyluk, Kody Dillman, Michael Hung, Teo Ulusoy, and Kendra Wannamaker---for the many fun (and sometimes long) game sessions.
Extra thanks to Kurtis for being the group's defacto DM and supplying modules, tokens, and other resources.
Gratitude is also owed to the other members of the iLab---Sandeep George, David Ledo, Claire Mikalauskas, Terrance Mok, Lora Oehlberg, Jessi Stark, Tiff Wun, and everyone else that welcomed me to your space as the ``graphics exchange student.''


In addition to everything already listed, certain individuals deserve extra recognition.
Mia, thank you for being one of my best friends, both before and throughout this degree.
Many distracting conversations and coffee breaks (especially the ones where only one of us would get coffee) were only possible because of you.
Lakin, thank you for the countless ways you have supported and mentored me in multiple facets of life.
Additional thanks for sparking my love of whisky and recommending beers I only hate \textit{a little bit}.
John Hall, I've known you for much longer than anyone else on the 6th floor.
Thank you for the many insightful discussions and always being willing to answer a quick question.
Kat, thank you for being an incredible friend and showing me around Boston.
You're an honourary Canadian in my books; best of luck to you in whichever city you end up settling down in.
Tiff, thank you for being the rock that anchored me during the most challenging time of my degree.
I don't know if I would have made it through without your friendship and support.


I want to thank all the students I had the privilege of TAing over the years.
Teaching was one of the most fulfilling and one of my favourite parts of grad school.
I am deeply grateful to have had the opportunity to pass on my knowledge and love of computer science and graphics to so many bright individuals.


Finally, thank you to my family for supporting me throughout (and for my entire life leading up to) this degree.
Thank you to my parents, Tim and Lisa, for bringing me into this world and raising me.
Thank you to my sisters, Hannah and Lily, for making sure life never gets too boring.
Last but not least, thank you to my dogs---Mitzie, Roxie, and the late Pebbles---for emotional support and many snuggles.


\begin{singlespace}
\newpage
\phantomsection
\tableofcontents
\pagestyle{plain}
\newpage
\phantomsection
\listoftables
\pagestyle{plain}
\newpage
\phantomsection
\listoffigures
\pagestyle{plain}
\clearpage
\clearpage          % otherwise tables will be numbered wrong
\end{singlespace}
