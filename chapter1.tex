\chapter{Introduction}
Large amounts of data being collected. Geospatial big data: data sets exceeding capacity of current computing systems~\cite{lee2015geospatial} Geospatial data has applications in urban planning, agriculture, education, natural disaster prediction and management, and countless other fields. Need tools to efficiently integrate, process, visualize, and analyze data in order to make informed decisions.

Traditional approach is GIS which allows datasets to be represented as map layers in coordinate system which acts as proxy for the earth. However different datasets need to be integrated, which exist in different formats and resolutions created by disparate organization. The conventional GIS pipeline which requires experts to clean, process, integrate, and distribute data is not sustainable in the face of geospatial big data. Furthermore, even once data has been integrated into common format/coordinate system, overlay operations are still computationally expensive. Can also imply precision that does not exist---points represented exactly when in reality there is uncertainty. 

Even with data integrated, GIS still has issues due to representing data on 2D map. Flattening comes with distortion and edges. Analyzing and understanding geospatial data is difficult when such distortions and cuts are present, especially for non-experts (buffering, geodesic flight paths, edge study). While physical maps have been historically favoured over globes for various practical reasons (manufacture, storage, scale), digital representations reduce or remove many drawbacks of globes. Computers/3D graphics can easily make the visualization. Storage is digital. Interactive systems allow zooming/panning to show any scale. Only problem not solved is not showing whole earth at once, however even this can be addressed with multi-view focus plus context techniques. Because of this, push towards Digital Earth where data is assigned and visualized on a 3D model of the Earth, typically approximated as either a sphere or ellipsoid.

To efficiently integrate data on such a 3D model, data can be assigned to an underlying discretization of the Earth into a set of non-overlapping cells (Discrete global grid). A hierarchy of such grids at successively finer resolutions allow different resolutions of data, and gives a built in representation for positional uncertainty (DGGS). 

DGGS have shown to be an effective tool for managing geospatial data on the surface of the Earth---many state of the art systems. Despite this, many modern sensors are generating 3D data (associated altitude) which traditional DGGS have no built in mechanism for. Instead, such data must be flattened with altitude stored only as an attribute. This slows down accessing data. This has motivated push towards volumetric, or 3D discrete global grids and grid systems to allow native support for such data.

Most 3D human activity and interest, and correspondingly geostatial data is in small region above and below surface of Earth (+= 10-500 km), and because of this most existing approaches to 3D grids have focused on this region. However, there a select processed such as seismic wave propagation and the magnetosphere the go much beyond (- ?? km and + 35000? km). Also high Earth orbit satellites. Because of this, need for grids that cover much larger range---high into atmosphere and all the way towards centre. 


\section{Problem Statement}

Despite recent push for 3D grids, still underdeveloped compared to traditional DGGS and GIS. Of these, most focus on small vertical range of data/grid support. The main goal of this thesis is to further develop the technologies that exist for 3D global grid systems that can properly support a large vertical range of data.

Simple approach for such a grid would be to embed Earth in 3D Euclidean grid, which many efficient and well tested techniques exist for creating. Such a grid does not respect spherical nature of Earth however. Cells not aligned with surface, low resolutions are very poor approximation, traversal along surface not straightforward. This type of grid should only be used for local small scale grids where curvature is minimal/negligible. For global grid, needs to be sphere based---Earth centric. 

The main challenge with such grids is the degeneration of cells towards the centre of the Earth due to the nature of spherical coordinates (figure)---cells become small, skinny, degenerate to triangular shaped/pyramids. This is not an issue for grids with small radial extent. This same effect happens with latitude longitude grids for similar reasons. Several works have used a special refinement strategy to address this issue, both for lat-long grids and 3D grids. (Further explained in Ch 2). Create grids we refer to as 'semiregular degenerate'. Such refinement strategies are the main tool used in this thesis for accomplishing the goal. For brevity, this class of refinement is simply referred to as semiregular refinement.  

Other desired properties of resulting grids we are concerned with are volume preservation, interoperability between 2D and 3D grids, and efficient encoding and decoding (integration) of data. 

\section{Methodology}

First approach is by modifying existing 3D DGGS known as SDOG to improve volume preservation.

This grid is good, but no interoperability between it and conventional DGGS.

Second approach is a general framework for extending DGGS to 3D. Good properties of such here.

With prismatoid grid, need to map between it and spherical 'real' space. Similar to projection for DGGS---needed for encoding and decoding. Use same principles to create mapping between conventional SDOG and modified one.

Develop more efficient encoding and decoding for these types of grids. Works for SDOG---and with the mapping---modified SDOG. Works for simple prismatoid, but not all. Discuss some methods for more complex prismatoid. Benchmark efficient vs simple for SDOG and modified SDOG.  

\section{Contribution}

Modification of SDOG grid to achieve better volume preservation
	- up to perfect volume preservation between all non-degenerate cells

Framework for extending any DGGS to 3D
	- all refinement factors
	- very good volume preservation properties. perfect for certain grids (excluding degenerate cells)

Mapping methods for: SDOG to modified SDOG; 3D prismatoid grid to spherical 3D grid

Efficient encoding/decoding algorithms for SDOG and other semiregular degenerate grids (constant time)
 	- benchmarking efficient vs hierarchical version for conventional SDOG and modified version

\section{Thesis Overview}

Chapter 2: background and related work
Chapter 3: modified refinement rules for SDOG
Chapter 4: grid extension framework
Chapter 5: mapping formulation for SDOG and prismatoid grids
Chapter 6: encoding and decoding algorithms and benchmarking
Chapter 7: conclusions and future work

