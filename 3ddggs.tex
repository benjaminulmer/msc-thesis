\chapter{3D Discrete Global Grid Systems} \label{chap:3ddggs}
In the previous chapter, we explored the DGGS as a tool for integrating multiple geospatial datasets at varying resolutions.
However, a DGGS has no inbuilt mechanism for handling the altitude(s) of geospatial data.
Instead, an extension of the data structure to the third dimension---the 3D DGGS---is needed to accommodate such 3D data natively.
This chapter argues for the need for 3D DGGS's in certain applications by exploring the downsides of a typical data flattening approach used to integrate 3D data with a 2D DGGS.
We then compare the two most common approaches used for creating a 3D DGGS: embedding and Earth-centric.
We argue for the Earth-centric approach, proposing semiregular refinement---the primary tool used in the remainder of this thesis---as a strategy to lessen the disadvantages of this approach.


\section{Data Flattening}
A DGGS provides a multiresolution partitioning of the surface of the Earth, but no partitioning in the radial dimension.
Therefore, any geospatial data with associate altitude must be flattened to the surface for integration with a DGGS, with the original altitude stored as a field of the data.
In applications where data \textit{does not} need to be filtered by altitude, and altitude is used no differently than any other attribute, this is an acceptable approach.
However, if data \textit{does} need to be accessed or otherwise distinguished by altitude, then this approach is problematic.
In this case, all data must be queried and searched to find those with the required altitude(s).
Furthermore, the lack of a hierarchy in the radial dimension means data at different radial resolutions must be integrated manually as opposed to using the structure of the DGGS itself to aid in this task.
Generally speaking, any benefits gained by using a DGGS are lost in the radial dimension; this is problematic for any application that uses the radial dimension in the same manner as (or more so than) surface ones.
Instead of managing altitude separately, the DGGS should be responsible for hierarchically partitioning altitude in addition to the surface of the Earth, which we call a 3D DGGS.


\section{Embedded Grids}
No ``up'' and ``down''.
Likewise, jagged surface traversal and poor approximation of Earth.
Does not respect spherical/multisphere nature of Earth.
Requires lat-long-alt to be converted to Cartesian (trig = expensive): some Earth-centric require as well, but unavoidable for embedded.
Less efficient possibly---cannot reuse surface stuff at different altitude.
Use figures to showcase these issues.


\section{Earth-Centric Grids}
Issue of degeneration similar to lat-long grids.
Use figures to showcase size difference and aspect ratio/compactness problem.


Same as regular grids but with altitude dimension.
Can be made from any of the above by extrusion method.
Lat-long -> 3D LLG -> GeoSOT3D.
Yin-yang -> 3D yin-yang.
Polyhedral -> frustum/prismatoid
\cite{yoo2019concept}
\cite{sun20153d}
\cite{yoshida2004application}
\cite{kageyama2005geodynamo}
\cite{tackley2008modelling}


Issue of radial degeneration
Methods have been proposed that use method similar to igloo in radial dimension.
DQG -> SDOG.
SSS 3DG.
Better prismatoid
\cite{yu2009sdog}
\cite{yu2012large-scale}
\cite{yu2012lithosphere}
\cite{gang2013sphere} 
\cite{wang2013global}


\subsection{Semiregular Refinement}
Several works have developed a particular style of grid to address this issue, both for latitude-longitude grids~\cite{leopardi2006partition, sun2008global} and 3D grids~\cite{yu2009sdog, gang2013sphere, wang2013global}\cite{others}.
These methods all have slight variations from one another, but the basic approach is the same.
During refinement, degenerate cells are merged with other degenerate cells to create larger cells. %TODO is this the best definition to use?
Despite still being degenerate, these merged cells end up being much more similar in size to the non-degenerate cells in the grid.
Figure~X illustrates this type of refinement in contrast to a regular refinement approach.
The terminology used in the literature to describe this type of refinement varies; in this thesis, we use the term \textit{semiregular degenerate} or for brevity just \textit{semiregular}.
A more in-depth explanation of this class of refinement methods is provided in Chapter~3.something.


Describe idea of method.
Useful for 3D and lat-long grids.
Cite works where type of refinement is used.
Allows for more uniform cells at cost of degenerate grid connectivity and slightly more complicated grid structure in general.
Main tool employed in this thesis.
