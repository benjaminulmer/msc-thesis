\chapter{Approaches for 3D DGGS's} \label{chap:3ddggs}
This chapter look at different method for 3D DGGS and why they are needed.
First look at why flattening not good enough solution to problem.
Explain drawbacks of embedded approach.
Challenges with Earth-Centric.
How semiregular refinement can be used to lessen these drawbacks.


\section{Data Flattening}
No partitioning in radial dimension.
This means slower data retrieval for specific altitudes, lack of spatial neighbourhood relationships, and other issues???
Furthermore, no hierarchy in radial dimension.
This means no automatic aggregation in this dimension.
Also loose any other benefits of grid in this dimension.
In general you simply loose all the benefits of a grid in the radial dimension.


\section{Embedded Grids}
No ``up'' and ``down''.
Likewise, jagged surface traversal and poor approximation of Earth.
Does not respect spherical/multisphere nature of Earth.
Requires lat-long-alt to be converted to Cartesian (trig = expensive): some Earth-centric require as well, but unavoidable for embedded.
Less efficient possibly---cannot reuse surface stuff at different altitude.
Use figures to showcase these issues.


\section{Earth-Centric Grids}
Issue of degeneration similar to lat-long grids.
Use figures to showcase size difference and aspect ratio/compactness problem.


\subsection{Semiregular Refinement}
Several works have developed a particular style of grid to address this issue, both for latitude-longitude grids~\cite{leopardi2006partition, sun2008global} and 3D grids~\cite{yu2009sdog, gang2013sphere, wang2013global}\cite{others}.
These methods all have slight variations from one another, but the basic approach is the same.
During refinement, degenerate cells are merged with other degenerate cells to create larger cells. %TODO is this the best definition to use?
Despite still being degenerate, these merged cells end up being much more similar in size to the non-degenerate cells in the grid.
Figure~X illustrates this type of refinement in contrast to a regular refinement approach.
The terminology used in the literature to describe this type of refinement varies; in this thesis, we use the term \textit{semiregular degenerate} or for brevity just \textit{semiregular}.
A more in-depth explanation of this class of refinement methods is provided in Chapter~3.something.



Describe idea of method.
Useful for 3D and lat-long grids.
Cite works where type of refinement is used.
Allows for more uniform cells at cost of degenerate grid connectivity and slightly more complicated grid structure in general.
Main tool employed in this thesis.
