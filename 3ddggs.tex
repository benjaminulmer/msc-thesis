\chapter{3D Discrete Global Grid Systems} \label{chap:3ddggs}
In the previous chapter, we explored the DGGS as a tool for integrating multiple geospatial datasets at varying resolutions.
However, a DGGS has no inbuilt mechanism for handling the altitude(s) of geospatial data.
Instead, an extension of the data structure to the third dimension---the 3D DGGS---is needed to accommodate such 3D data natively.
This chapter argues for the need for 3D DGGS's in certain applications by exploring the downsides of a typical data flattening approach used to integrate 3D data with a 2D DGGS.
We then compare the two most common approaches used for creating a 3D DGGS: embedding and Earth-centric.
We argue for the Earth-centric approach, proposing semiregular degenerate refinement---the primary tool used in the remainder of this thesis---as a strategy to lessen the disadvantages of this type of grid.


\section{Data Flattening} \label{chap:3:flatten}
A DGGS provides a multiresolution partitioning of the surface of the Earth, but no partitioning in the radial dimension.
Therefore, any geospatial data with associate altitude must be flattened to the surface for integration with a DGGS, with the original altitude stored as an attribute of the data.
In applications where data \textit{does not} need to be filtered by altitude, and altitude is used no differently than any other attribute, this is an acceptable approach.
However, if data \textit{does} need to be accessed or otherwise distinguished by altitude, then this approach is problematic.
In this case, all data of a cell must be queried and searched to find those with the required altitude(s).
Furthermore, the lack of a hierarchy in the radial dimension means data at different radial resolutions must be integrated manually as opposed to using the structure of the DGGS itself to aid in this task.
Generally speaking, any benefits gained by using a DGGS are lost in the radial dimension; this is problematic for any application that uses the radial dimension as frequently as surface ones.
In these applications, instead of managing altitude separately, the DGGS should be responsible for hierarchically partitioning altitude in addition to the surface of the Earth.
We call this a 3D DGGS.


\section{Embedded Grids} \label{chap:3:embedded}
Now treating the Earth as a fully 3D entity, it seems natural to use a conventional Euclidean space partitioning data structure such as a voxel grid or octree to manage geospatial data.
We refer to this approach as an embedding one, creating an embedded 3D DGGS.
While an embedding approach is straightforward and leverages existing data structures and algorithms, it disregards the actual shape of the Earth.
Regardless of how finely refined cells are, they can only approximate the underlying representation of the Earth: a sphere or ellipsoid (Figure~\ref{fig:embedded}).
Furthermore, cells do not align with the surface of the Earth, which means cells have no consistent orientation of what direction represents an increasing or decreasing altitude, or up and down.
Because of this, traversing along the surface of the Earth becomes a non-straightforward operation that requires a ``zig-zagging'' path.
Due to the multi-sphere structure of the Earth, these challenges also manifest when dealing with different subterranean and atmospheric layers.


\begin{figure}[h]
	\centering
	\includegraphics[width=0.6\textwidth]{embedded.pdf}
	\caption[A Circle Embedded in a Regular Cartesian Grid]{
		A circle embedded in a regular 2D Cartesian grid.
		Cells do not align with the circle, which means the cells that represent the surface of the circle (blue) contain varying amounts of said circle.
		Regardless of how finely refined such a grid is, it will only ever approximate the surface.
		The same applies to a sphere or ellipsoid embedded in a 3D grid
	}
	\label{fig:embedded}
\end{figure}


Despite the issues present with an embedded grid, for small-scale regions with negligible curvature, these issues also become negligible.
Because of this, embedded grids are frequently used for small-scale objects and phenomena in use cases such as surface building, rock engineering, and reservoir modelling. [citations]
However, the challenges presented by an embedding approach outweigh their benefits in the context of a 3D DGGS, which needs to have global coverage.


The inherent issues with an embedded 3D DGGS stem from the fact that they are based on Cartesian coordinates as opposed to spherical ones.
Thus far, we have described 3D geospatial data as data with an associated altitude in addition to its surface coordinates, which is a spherical representation of its location.
Due to the mostly spherical shape of Earth, a spherical coordinate system is a natural choice, and a 3D DGGS should respect this coordinate system.
We call a 3D DGGS based on, or sharing similar properties to, a spherical coordinate system Earth-centric.


\section{Earth-Centric Grids} \label{chap:3:earthCentric}
The most fundamental Earth-centric 3D grid is a direct extension of an LLG to the third dimension.
In addition to diving space by latitude and longitude, a 3D LLG also divides space by equal steps in the radial dimension.
Due to the simplicity of its construction, this type of grid has seen use in global crust modelling~\cite{bassin2000current} and exploring P-wave velocity~\cite{zhao2004global}, among other applications.
The GeoSOT3D grid, proposed by Sun and Cheng, is a variation of the standard 3D LLG where latitude and longitude have been extended to larger virtual spaces to improve the efficiency of grid encoding and decoding~\cite{sun20153d} and has been used to increase the efficiency of aircraft and UAV collision detection~\cite{miao2019low, zhai2019collision}.


Similar radial extensions are also possible for many of the DGG's and DGGS's explored in Chapter~\ref{chap:2:DGG}.
Being composed of two component LLG's, an extension of the Yin-Yang grid to 3D can be done in the same manner as above.
These 3D Yin-Yang grids have been used extensively for geodynamo and mantle convection simulation~\cite{yoshida2004application, kageyama2005geodynamo, tackley2008modelling}.
Extensions of polyhedron based DGGS's to 3D have also been done.
An approach by Xie et al. stacks duplicates of a DGGS at increasing radii to create a 3D global grid for interactive volumetric ray tracing of 3D Earth data~\cite{xie2013interactive}.
While this approach is appropriate for the intended use, it is not technically a 3D DGGS, as the radial component of the grid is separate from the cell hierarchy.
In order to incorporate the radial component of the grid with the cell hierarchy, Sirdeshmukh et al. create a 3D (or 4D) DGGS by placing 3D (or 4D) hypercubes on the faces of the DGGS polyhedron~\cite{sirdeshmukh2019utilizing}.
These hypercubes are then refined and traversed with a space-filling curve to create the 3D (or 4D) DGGS, which is used by the authors to encode point cloud data.


Beyond the singularities at the poles, spherical coordinates also have a singularity at the centre of the sphere.
This central singularity causes the same issues near the centre of the grid as seen near the poles of an LLG---namely reduced compactness, a high valence vertex, and an unbounded difference in cell \textit{volume}.
None of the above methods address this central singularity, as these issues are mostly negligible for applications with small ranges of altitudes.
However, in order to accommodate large altitude ranges (as is the goal of this thesis), these issues are significant and should be addressed.


Similar to how the DQG addresses polar singularities in an LLG, a similar approach can be used for the central singularity in a 3D grid.
A direct extension of the DQG to 3D, known as SDOG, was proposed by Yu and Wu~\cite{yu2009sdog}.
Just as the DQG utilizes a modified quadtree refinement, SDOG extends this refinement to 3D, creating a modified octree refinement.
The resulting cells are relatively uniform in both size and shape, having a bounded maximum difference in volume.
Because of these properties, SDOG has been used for the modelling of large-scale geospatial objects~\cite{yu2012large-scale} and multi-scale visualization of the lithosphere~\cite{yu2012lithosphere}.
We provide a more detailed explanation of SDOG and its specific refinement method in Chapter~\ref{chap:sdog}.
Another method similar to SDOG is the Sphere Shell Space 3D Grid, which uses a similarly modified octree refinement of cells but is based on multiple sphere shells as opposed to a single sphere~\cite{gang2013sphere}.
Likewise, Wang et al. also apply a similar process to a great circle arc quaternary triangular mesh (QTM) to create a 3D DGGS~\cite{wang2013global}.
These methods all use a similar refinement strategy to handle radial degeneration in the grid; however, the terminology used to describe this type of refinement is inconsistent between works.
In this thesis, we propose the term \textit{semiregular degenerate refinement} to refer to this class of methods.


\subsection{Semiregular Degenerate Refinement} \label{chap:3:semiregDegen}
A regular grid refinement is one that results in regular connectivity in the output domain.
For triangular grids, this means each vertex has a valence of exactly six; for quadrilateral grids, a valence of four; and for hexagonal grids, a valence of three.


\begin{figure}[h]
	\centering
	\includegraphics[width=0.7\textwidth]{semireg-degen.pdf}
	\caption[Comparison of Regular and Semiregular Degenerate Refinement]{
		A demonstration of regular and semiregular degenerate refinement schemes for quadrilaterals.
		Left: regular refinement applied to a quadrilateral domain.
		Centre: regular refinement applied to a triangular domain.
		Right: semiregular degenerate refinement applied to a triangular domain.
		The bottom row shows regions of regular connectivity in the resulting grids; note how the semiregular degenerate scheme results in several regular regions, but the grid itself is not fully regular
	}
	\label{fig:semireg-degen}
\end{figure}


Looking specifically at regular quadrilateral refinement, one way to achieve this is by splitting each input cell into four output ones with two intersecting straight lines, as shown in Figure~\ref{fig:semireg-degen}~left.
As we have seen, applying such a regular refinement to a degenerate (triangular) starting cell---such as spherical octant or a slice going from the surface of a sphere to the centre---still results in a regular grid, but the compactness and size of cells are affected (Figure~\ref{fig:semireg-degen}~centre).
To lessen this effect, the splitting edge that would normally intersect the singularity can be stopped at the other edge, as demonstrated in Figure~\ref{fig:semireg-degen}~right.
When applying this scheme recursively, only the degenerate (triangular) cells use the special refinement, and non-degenerate (quadrilateral) cells use regular refinement.
As illustrated in Figure~\ref{fig:semireg-degen}~bottom, this refinement method results in a grid with regular regions, but the grid as a whole is not itself regular.
Hence, we call this semiregular degenerate refinement.
The same principles apply in 3D for hexahedral grids; however, there are instead three intersecting \textit{faces} used in refinement.
In general, we define a semiregular degenerate refinement to be one in which one or more splitting geometries (curves in 2D and surfaces in 3D) are stopped by another splitting geometry in the direction toward a singularity where, if not stopped, would otherwise intersect said singularity.
Such a definition describes the methods used in DQG, SDOG, Sphere Shell Space 3D Grid, the great circle arc QTM extension, and those used in Chapter~\ref{chap:extension} of this thesis.


A semiregular degenerate refinement allows for more uniformly sized and compact cells as compared to a regular refinement applied to the same domain.
The main drawback of this class of refinement is the degenerate connectivity introduced between the different semiregular regions of the grid.
In a non-degenerate grid, each cell will have precisely one neighbour along each edge, or no neighbours if the cell is on a boundary of the grid.
In 3D, the same applies but with faces instead of edges.
Grids resulting from semiregular degenerate refinement violate this property, with some cells having multiple neighbours along an edge/face.
Such degenerate connectivity introduces challenges when propagating values across boundaries of cells and calculating values at vertices, as is needed for finite element methods.
There has been work to address this challenge for conventional octrees~\cite{braun2008douar}, which have a similar issue between regions of the grid at different levels of refinement; similar methods could likely be applied to semiregular degenerate grids as well.
Considering the tradeoffs of this approach, we believe the benefits outweigh the disadvantages in the context of a 3D DGGS.
As such, semiregular degenerate refinement is the primary tool used in the remainder of this thesis to achieve the goals set out in Chapter~\ref{chap:1:goals}.
