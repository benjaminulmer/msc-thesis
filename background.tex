\chapter{Background and Related Work} \label{chap:background}
Providing a complete history of map and globe making (cartography) is much beyond both the scope and ability of this thesis.
Instead, we provide a brief background to illuminate how these disciplines influenced the developments of Digital Earth systems.
We discuss the different approaches for a Digital Earth, starting with traditional GIS and then discussing techniques used for globe based systems and analysis.
We also provide background and related work on discrete global grids and grid systems, looking at different methods for constructing both 2D and 3D versions of these data structures.


\section{Traditional Maps and Globes}
Cartography, the study and practice of making maps, is a mature and diverse discipline.
While there is no agreement on what is the earliest known map, several examples dating back to the 4th millennium BCE exist.
World maps are not quite as old; however, there are surviving specimens from 9th century BCE Babylonia.


In the 3rd century BCE, the ancient Greeks established that the Earth is spherical; with this came the first globes of Earth.
Despite this discovery, maps still were---and continue to be---the primary method for representing the Earth and geospatial data~\cite{hruby20182000}.
Compared to maps, globes are more expensive to manufacture and more difficult to store, transport, and make measurements on.
Maps can also represent the entire Earth at once and easily accommodate any scale.
The benefits of a flat map do not come without their disadvantages, however.
Producing a map requires flattening the spherical Earth to a flat plane, an operation known as a map projection (or simply projection), which introduces inevitable distortions and map edges.
These issues are not present on globes, and because of this, they provide a more accurate representation of the Earth than is possible with any map.


\subsection{Map Projections}
Despite the fundamental challenges associated with map projections, the many benefits of maps over globes have motivated the use and study of these operations.
Distortions in a projection (or more generally in any mapping) are measured by their effect on angles, areas, and distances.
While eliminating all distortion is impossible, specialized projection methods exist that can remove certain types of distortion.
Commonly used conformal [] and area-preserving [] projections preserve angles and areas, respectively.
Isometric projections---which preserve distances---also remove angle and area distortion, and therefore do not exist between the sphere and the plane.
However, projections that preserve only certain distances, such as all those to a certain point, are possible. []
Alternatively, some methods aim for a balanced trade-off between the different types of distortion, as opposed to eliminating a specific class. []

Since distortion cannot be completely removed,

Map edges introduced by projection can also impact people's understanding 


Map projections introduce inevitable distortion and map edges
\cite{snyder1987map}
\cite{snyder1997flattening}


Come with inevitable distortion.
Certain projections can eliminate certain types of distortion i.e. equal area, conformal, etc.
Despite this, cannot completely remove all distortion.
Cannot perfectly preserve all distances.
\cite{mathematics paper on distortion}
\cite{planar area preserving}
\cite{planar conformal}


Map edges can also introduce issues with understanding maps.
\cite{hennerdal2015beyond}
\cite{hruby2016journey}


\subsubsection{Polyhedral Projections}
Using more planes can reduce distortion.
Method known and used since at least \textbf{date}.
Drawback of introduces additional map edges when flattened.
Benefit of can be folded back into pseudo globe.
\cite{bradley1946equal}
\cite{snyder1992equal}
\cite{van2006slice}
\cite{rocsca2011uniform}
\cite{rocsca2012area}
\cite{holhocs2014octahedral}
\cite{harrison2011optimization}


\section{Digital Earth Systems}
Many existing systems for creating Digital Earth.
Many different flavours for different tasks.
Maps, streetview, Earth, GIS, etc.
\cite{mahdavi2015survey}
\cite{alderson2020digital}


\subsection{GIS}
Digital Earth based on principles and techniques of traditional geography.
Based on 2D maps.
Use map ``layers'' to represent different data sets.
Layers can be combined and used for different analysis, visualizations, etc.
Distortion can cause errors in this analysis; must use spherical analysis sometimes to avoid this.
Most systems require some amount of user expertise to clean, integrate, process data.
\cite{antenucci1991geographic}
\cite{foresman1998history}


\subsection{Globe Based Digital Earth}
Use globe as reference model as opposed to flat map.
Two approaches:
(1) use underlying flat representation and map to globe for visualization.
Does not solve issues of distortion for analysis purposes, only visualization.
\cite{goodchild2018reimagining}


Other approach is to do analysis directly in spherical or ellipsoidal space.
More computational expensive than planar equivalents, but much more accurate.
Ellipsoidal is challenging due to issues of elliptic integrals; various well known techniques for sphere however.
Raskin paper, Troy's papers, points on sphere paper, maybe others.
\cite{raskin1994spatial}
\cite{bahrdt2017rational}
\cite{alderson2016multiresolution}
\cite{alderson2019multiscale}
\cite{alderson2018offsetting}


\section{Discrete Global Grids}
Spatial partitioning structures have long been used in computer graphics for managing spatial information.
Reasons and benefits of such.
Same techniques can be used for geospatial data.
Uses of global grids varies.
From structure for speeding up queries to foundation for entire Digital Earth system (pyxis/ggs) as alternative to GIS system.
However DGG can also be used in tandem with traditional GIS technologies.
Many different techniques exist for creating such grids (and by extension grid systems).
\cite{sahr1998discrete}
\cite{sahr2003geodesic}


\subsection{Discrete Global Grid Systems}
A discrete global grid only represents Earth at one spatial resolution.
Geospatial data comes in large range of resolutions.
This can be addressed by creating a hierarchy of DGG's.
This hierarchy is what we call a DGGS.


\subsubsection{Refinement}
Start with initial discretization and then refine.
Discretization can be any DGG.


The refinement method of a DGGS defines the process by which a set of fine cells are produced from a set of coarse cells.
This should be done in a consistent manner such that it can be applied successively to create increasingly fine discretizations of the Earth.
These refinement schemes are classified by their input cell shape(s), which are given by the initial discretization; output cell shape(s), which are usually the same as the input; and their refinement factor, which is simply the ratio between the number of coarse cells and fine cells.
For example, triangle 1-to-4 (1:4) refinement produces four triangle cells for each triangle in the set of coarse cells.
The refinement factor determines how quickly the resolution of the grid increases with each application of the refinement method.


Refinement vs. subdivision: Both terms appear in the literature.
In computer graphics, subdivision typically refers to subdivision surfaces which involve repositioning vertices.
To avoid confusion, this thesis will use the term refinement to refer to the process of creating a set of children cells from parent cells.


\subsection{Spherical Grids}
Lat-long grids.
Semiregular/igloo lat-long grids (DQG).
Small circle arc grids.
Yin-yang grid.
\cite{leopardi2006partition}
\cite{sun2008global}
\cite{song2002developing}
\cite{kageyama2004yin-yang}


\subsection{Polyhedral Grids}
Main idea is to use polyhedron as approximation for sphere.
Polyhedral projection as described earlier maps spherical domain to polyhedral one.
Avoids polar degeneracies.
Area preserving projections allow equal area cells without significantly reducing compactness.
\cite{fekete1990sphere}
\cite{dutton1996encoding}
\cite{gorski2005healpix}
\cite{holhocs2014octahedral}
\cite{mahdavi2013one} % TODO ???
\cite{mahdavi2015hexagonal} % TODO ???


\subsection{3D Grids}
Same as regular grids but with altitude dimension.
Can be made from any of the above by extrusion method.
Lat-long -> 3D LLG -> GeoSOT3D.
Yin-yang -> 3D yin-yang.
Polyhedral -> frustum/prismatoid.
\cite{yoo2019concept}
\cite{sun20153d}
\cite{yoshida2004application}
\cite{kageyama2005geodynamo}
\cite{tackley2008modelling}


Issue of radial degeneration---see Chapter 3.something
Methods have been proposed that use method similar to igloo in radial dimension.
DQG -> SDOG.
SSS 3DG.
\cite{yu2009sdog}
\cite{yu2012large-scale}
\cite{yu2012lithosphere}
\cite{gang2013sphere} 
\cite{wang2013global}


\subsection{Grid Encoding and Decoding}
Need to map data/features to grid cells.
Most simple encoding is for points: find cell that contains point.
Inverse also needed: take cell and get geometry on reference model.
Call these encoding and decoding.
As part of this, need way to refer to cells uniquely.
\cite{du2018duality}


\subsubsection{Indexing Scheme} \label{sec::dggs:indexing}
For grid and cell geometries to be useful as a DGGS, a mechanism for assigning a unique index to each cell is needed.
Earth data is assigned to cells using these indices, which allows for efficient retrieval of all data associated with a specific cell or region of the Earth.
Thus, in order to be able to insert new data into the grid efficiently, it is important to be able to quickly determine the cell (and associated index) that contains a point on the Earth after it has been projected.
Likewise, given a cell index, it is important to be able to determine the geometry of the corresponding cell in the grid.
This geometry can then be inverse projected to find the region of the Earth associated with the cell.
These indices also allow for certain spatial relationship queries to be defined, such as the retrieval of neighbouring cells, and can be used to navigate up and down the hierarchy of the grid via parent and child relationships between cells in different resolutions.
\cite{yu2009coding}
\cite{vince2006indexing}
\cite{tong2013efficient}
\cite{mahdavi2015categorization}


\section{Summary}
Maps and globes have been around for thousands of years.
With advent of computer, GIS was born which carried over conventions of traditional cartography.
Recently, global grids have become popular as a replacement/augmentation of GIS.
Many different approached for DGG and DGGS.
3D DGGS are the next step of this evolution; techniques exist but still underdeveloped comparatively.
